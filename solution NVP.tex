\documentclass{article}
\usepackage{amsmath,amssymb}
\DeclareMathOperator{\E}{\mathbb{E}}
\usepackage[utf8]{inputenc}
\usepackage[english]{babel}
\usepackage{biblatex}
\addbibresource{novel.bib}
\usepackage{graphicx}
\usepackage{indentfirst}
\usepackage{physics}
\usepackage{refcheck}
\usepackage{units}

\begin{document}
\section{Solution of classical newsvendor problem} 

The expected profit can be expressed as:
\[
    \E[\pi(Q)]=\int_{0}^{Q}[py-vQ+c_h(Q-y)]f_{Y}(y)dy+\int_{Q}^{\infty}[pQ-vQ+c_s(y-Q)]f_{Y}(y)dy
\]
By reforming, we have:
\[
    \E[\pi(Q)]=(p-c_h)\mu-(v-c_h)Q-(p-c_h-c_s)\int_{Q}^{\infty}(y-Q)f_{Y}(y)dy
\]
Take derivative with respect to $Q$:
\[
    \pdv{\E[\pi(Q)]}{Q}=-(v-c_h)+(p-c_h-c_s)\{-Q[-f_{Y}(Q)]+[1-F_{Y}(Q)]+Q[-F_{Y}^{'}(Q)]\}
\]
Use the first order condition:
\[
    -(v-c_h)+(p-c_h-c_s)[1-F_{Y}(Q)]\equiv0
\]
Accompany with convexity:
\[
    \pdv[2]{\E[\pi(Q)]}{Q}=(p-c_h-c_s)[-f_{Y}(Q)]
\]
We have:
\[
    F_{Y}(Q^{*})=1-\frac{v-c_h}{p-c_h-c_s} =\frac{p-v-c_s}{p-c_h-c_s}
\]
That is:
\[
    Q^*=F_{Y}^{-1}\left(\frac{p-v-c_s}{p-c_h-c_s}\right)
\]

\section{Solution of nonlinear newsvendor problem} \label{app:A}

As before, we let $f(y)$ denote the probability density function of the demand and $\Pi(Q)$ denote the expected value of $\pi(Q,y)$. We have:
\begin{eqnarray*}
\Pi(Q) = && -V(Q)
+ \int_0^\infty P(Q,y) f(y) \, dy \\
&& - \int_0^Q C_h(Q,y) f(y) \, dy 
 - \int_Q^\infty C_s(Q,y) f(y) \, dy.
\end{eqnarray*}
Assuming that the functions $P$, $V$, $C_h$ and $C_s$ are everywhere differentiable, we have:
\[
\pdv{\Pi}{Q} \, = \, -\pdv{V}{Q} \, + \,
\int_0^\infty \, \pdv{P}{Q} \, f(y) \, dy
\, - \,
\int_0^Q \, \pdv{C_h}{Q} \, f(y) \, dy
\, - \,
\int_Q^\infty \, \pdv{C_s}{Q} \, f(y) \, dy.
\]
If we make the additional assumption that $\Pi$ is a strictly concave function of $Q$, then we can find the unique local maximum by setting $\pdv{\Pi}{Q}$ to zero.

With the condition of $\pdv{\Pi}{Q}{y}=0$, we have:
\[
    \pdv{P}{Q}-\pdv{V}{Q}+\pdv{C_s}{Q}=\big(\pdv{P}{Q}-\pdv{C_h}{Q}+\pdv{C_s}{Q}\big)F(Q)
\]
Therefore, the following equation holds:
\[
    Q^*=F^{-1}\left(\frac{P_{(Q)}-V_{(Q)}+C_{s(Q)}}{P_{(Q)}-C_{h(Q)}+C_{s(Q)}}\right),
\]
where $P_{(Q)}=\pdv{P}{Q}$. When the condition do not hold, numerical approaches are required to solve the optimal order. No matter the approaches, we always denote
\[
    Q^*=F^{-1}(g)
\]
to be the optimal order in a quantile form. The quantile is now a function instead of constant. In general, $Q^*$ is the $g$\textsuperscript{th} quantile of demand distribution and $g=\tau$ is a constant in NVP.
\end{document}